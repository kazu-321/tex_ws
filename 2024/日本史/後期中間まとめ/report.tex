\RequirePackage{plautopatch}
\documentclass[a4paper,11pt]{jsarticle}
\usepackage{color}
\usepackage{subcaption}
\usepackage{listings,jvlisting}
\usepackage{url}
\usepackage{color}
\usepackage{soul}
\usepackage{pxrubrica}
% 数式
\usepackage{amsmath,amsfonts}
\usepackage{bm}
% 画像
\usepackage[dvipdfmx]{graphicx}
\lstset{
  stringstyle={	tfamily},
  commentstyle={	tfamily},
  basicstyle={	tfamily},
  columns=fixed,
  frame={tb},
  breaklines=true,
  columns=[l]{fullflexible},
  numbers=left,%行数を表示したければonにする
  numberstyle={\scriptsize},
  xrightmargin=0em,
  xleftmargin=3em,
  stepnumber=1,
  numbersep=1em,
  tabsize=2,
  lineskip=-0.5ex,
  backgroundcolor=\color{white}
}
\usepackage{geometry}
\geometry{left=25mm,right=25mm,top=20mm,bottom=20mm}
\usepackage{enumitem}
\usepackage{float}
\usepackage{tikz}
\usetikzlibrary{shapes.geometric, arrows,positioning}
\tikzstyle{startstop} = [rectangle, rounded corners, minimum width=3cm, minimum height=1cm,text centered, draw=black, fill=red!20]
\tikzstyle{process0} = [rectangle, minimum width=3cm, minimum height=1cm, text centered, draw=black, fill=red!20]
\tikzstyle{process1} = [rectangle, minimum width=3cm, minimum height=1cm, text centered, draw=black, fill=blue!20]
\tikzstyle{process2} = [rectangle, minimum width=3cm, minimum height=1cm, text centered, draw=black, fill=orange!30]
\tikzstyle{process3} = [rectangle, minimum width=3cm, minimum height=1cm, text centered, draw=black, fill=green!30]
\tikzstyle{arrow} = [thick,->,>=stealth]
\tikzstyle{io} = [trapezium, trapezium left angle=70, trapezium right angle=110, minimum width=3cm, minimum height=1cm, text centered, draw=black, fill=pink!40]

\newcommand{\important}[2]{\textbf{\colorbox{yellow}{\ruby[g]{#1}{#2}}}}



\title{日本史後期中間まとめ}
\date{\empty}
\author{橋本 千聡}
\begin{document}
\maketitle
\section*{テスト範囲}
\begin{itemize}
  \item 教科書P167~P218
  \item プリント
  \begin{itemize}
    \item 都市と商業・手工業
    \item 幕政の改革
    \item 幕府の衰退と近代への道
  \end{itemize}
\end{itemize}

\section*{出題形式}
\begin{itemize}
  \item 語群問題(60点)
  \item 正誤問題(30点)
  \item 筆記問題(10点)
\end{itemize}

\newpage

\section*{都市と商業・手工業}
\subsection*{村と百姓}
\subsubsection*{(1) 村の運営}
\begin{itemize}
  \item \important{本百姓}{ほんびゃくしょう}: 検地帳に登録、年貢・諸役を負担、村政に参加
  \item \important{村方三役}{むらかたさんやく} (名主・組頭・百姓代): 村政の指導者、本百姓から選ばれることが多い
\end{itemize}

\subsubsection*{(2) 百姓の負担}
\begin{itemize}
  \item \important{本途物成}{ほんとものなり} (本年貢): 田畑・屋敷地に課税、米納が原則(4〜5割)
  \item \important{小物成}{こものなり}: 山野河海の利用、農業以外の副業に課税
\end{itemize}

\subsubsection*{(3) 百姓の統制}
\begin{itemize}
  \item \important{村請制}{むらうけせい}: 年貢・諸役の負担を村全体に割り当てる制度
  \item \important{五人組}{ごにんぐみ}: 年貢納入、犯罪防止に対する連帯責任制度
\end{itemize}

\subsection*{幕政の安定}
\subsubsection*{4代将軍 \important{徳川}{}\important{家綱}{いえつな}〔在 1651〜1680〕}
\begin{itemize}
  \item 武断政治から\important{文治政治}{ぶんちせいじ}への転換
  \item 末期養子の禁の緩和: 50歳未満の大名が死の間際にとる養子を容認
  \item 殉死の禁止: 主人の死に殉じる戦国の遺風を廃止
\end{itemize}

\subsubsection*{5代将軍 \important{徳川}{}\important{綱吉}{つなよし}〔在 1680〜1709〕}
\begin{itemize}
  \item \important{文治主義}{ぶんちしゅぎ}の徹底
  \item \important{武家諸法度}{ぶけしょはっと}(天和令)を発布(1683): 「文武忠孝を励まし、礼儀を正すべき事」→以前は「文武弓馬」だった
  \item 儒学の奨励: 朱子学者・木下順庵に学び、湯島聖堂を建立
\end{itemize}

\subsubsection*{6代将軍 \important{徳川}{}\important{家宣}{いえのぶ}〔在 1709〜1712〕}
\begin{itemize}
  \item 朱子学者・\important{新井白石}{あらいはくせき}を登用し、政治を刷新
  \item 7代将軍 徳川家継〔在 1713〜1716〕の治世を含めて、正徳の治という
  \item 朝幕関係の改善: 閑院宮家の創設
  \item 正徳小判鋳造: 貨幣価値を上げ、物価の抑制をねらう
\end{itemize}

\newpage

\subsection*{経済の発展}
\subsubsection*{農具}
\begin{itemize}
  \item 備中鍬、千歯扱、唐箕
\end{itemize}

\subsubsection*{肥料}
\begin{itemize}
  \item 干鰯・油粕などの金肥
\end{itemize}

\subsubsection*{産業}
\begin{itemize}
  \item 入浜塩田、西陣織
\end{itemize}

\subsubsection*{交通}
\begin{itemize}
  \item \important{五街道}{ごかいどう} (1東海道、2中山道、3甲州道中、4日光道中、5奥州道中)
  \item \important{菱垣廻船}{ひがきかいせん}と\important{樽廻船}{たるかいせん}
  \item \important{東廻り海運}{ひがしまわりかいうん}(東北日本海側~津軽海峡~那珂湊~江戸)
  \item \important{西廻り海運}{にしまわりかいうん}(東北日本海側~下関~大坂)
\end{itemize}

\subsubsection*{貨幣}
\begin{itemize}
  \item \important{三貨}{さんか} (金・銀・銭)を幕府が鋳造 「江戸の金遣い、大坂の銀遣い」
\end{itemize}

\subsubsection*{天下の台所}
\begin{itemize}
  \item 大坂には諸藩の蔵屋敷が密集、蔵元・掛屋が活躍
  \item \important{三都}{さんと} (江戸・大坂・京都)のうちの一つ
\end{itemize}

\newpage

\section*{幕政の改革}
\subsection*{享保の改革}
\begin{itemize}
  \item 8代将軍 \important{徳川}{}\important{吉宗}{よしむね}〔在 1716~1745、もと紀伊藩主〕
  \item \important{御用取次}{ごようとりつぎ}: 将軍の意志を幕政に反映させる
\end{itemize}

\subsubsection*{享保の改革}
\begin{itemize}
  \item \important{相対済し令}{あいたいすましれい}: 金公事の訴えを幕府は受理しない→当事者間で解決させる
  \item \important{上げ米}{あげまい}: 大名が1万石につき100石の米を上納→参勤交代の在府期間を半年に
  \item \important{目安箱}{めやすばこ}の設置: 評定所に設置し、庶民の投書により\important{小石川養生所}{こいしかわようじょうしょ}を設立
  \item \important{公事方御定書}{くじかたおさだめがき}の制定: 司法の基準
\end{itemize}

\subsection*{一揆と打ちこわし}
\begin{itemize}
  \item 大規模なものは、\important{享保の飢饉}{きょうほうのききん}、\important{天明の飢饉}{てんめいのききん}のとき発生
\end{itemize}

\subsubsection*{百姓一揆}
\begin{itemize}
  \item \important{代表越訴型一揆}{だいひょうおっそがたいっき}(17世紀後半): 代表者が領主に直訴する形態
  \item \important{惣百姓一揆}{そうひゃくしょういっき}(17世紀末): 村全体の百姓がおこす大規模な一揆
\end{itemize}

\subsubsection*{打ちこわし}
\begin{itemize}
  \item 町人・農民が富商・金融業者・米問屋などを襲撃
\end{itemize}

\subsection*{田沼時代}
\begin{itemize}
  \item 10代将軍 \important{徳川}{}\important{家治}{いえはる}〔在 1760~1786〕、老中\important{田沼意次}{たぬまおきつぐ}の時代
  \item 意次は9代将軍 徳川家重〔在 1745~1760〕のとき御用取次をつとめた
\end{itemize}

\subsubsection*{田沼意次の政策}
\begin{itemize}
  \item \important{株仲間}{かぶなかま}の(積極的)公認: 営業税の\important{運上}{うんじょう}・\important{冥加}{みょうが}の増収をねらう
  \item \important{南鐐二朱銀}{なんりょうにしゅぎん}など計数銀貨を鋳造: 金を中心とする貨幣制度に一本化をめざす
  \item \important{蝦夷地}{えぞち}の開発・ロシアとの交易の調査: \important{最上徳内}{もがみとくない}らを蝦夷地に派遣
\end{itemize}


\newpage

\section*{幕府の衰退と近代への道}
\subsection*{寛政の改革}
寛政の改革: 11代将軍 \important{徳川}{}\important{家斉}{いえなり}〔在 1787〜1837〕のとき、老中\important{松平}{まつだいら}\important{定信}{さだのぶ}による改革
\begin{itemize}
  \item \important{囲米}{かこいまい}: 飢饉に備えて社倉・義倉に米穀を蓄えさせる
  \item \important{七分積金}{しちぶつみきん}: 町費の節約分の7割を町会所で運用・積立→飢饉・災害に備える
  \item \important{寛政異学の禁}{かんせいいがくのきん}: 朱子学を正学とし朱子学以外の講義・研究を禁止 @聖堂学問所
\end{itemize}

\subsection*{鎖国の動揺と大塩の乱}
\begin{itemize}
  \item 1792年: ロシアの\important{ラクスマン}{}が根室に来航→日本人漂流民を届け、通商を要求
  \item 1808年: \important{フェートン号事件}{ふぇーとんごうじけん} = イギリス軍艦が長崎に乱入
  \item 1825年: \important{異国船打払令}{いこくせんうちはらいれい}を発令→従来の薪水・食糧の給与を撤回
  \item 1832〜33年: \important{天保の飢饉}{てんぽうのききん}→百姓一揆・打ちこわしの続発
  \item 1837年: 大坂町奉行所の元与力・\important{大塩平八郎}{おおしおへいはちろう}が蜂起
\end{itemize}

\subsection*{天保の改革}
天保の改革: 12代将軍 \important{徳川}{}\important{家慶}{いえよし}〔在 1837〜1853〕のとき、老中\important{水野忠邦}{みずのただくに}による改革
\begin{itemize}
  \item \important{人返しの法}{ひとがえしのほう}: 江戸に流入した貧民の帰郷を強制
  \item \important{株仲間の解散}{かぶなかまのかいさん}: 物価の引き下げがねらい
  \item \important{上知令}{じょうちれい}: 江戸・大坂周辺を直轄化→ことごとく失敗に終わる
\end{itemize}

\subsection*{経済の変化と雄藩の浮上}
\begin{itemize}
  \item \important{工場制手工業}{こうじょうせいしゅこうぎょう} (マニュファクチュア): 商人が奉公人を工場に集め、分業と協業で手工業品を生産
  \item 大坂周辺・尾張→綿織物、桐生・足利→絹織物
  \item \important{薩長土肥}{さっちょうどひ}: 薩摩・長州・土佐・肥前などの大藩が改革に成功、水戸は失敗
\end{itemize}

\end{document}