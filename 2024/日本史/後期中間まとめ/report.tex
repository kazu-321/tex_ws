\RequirePackage{plautopatch}
\documentclass[a4paper,11pt]{jsarticle}
\usepackage{color}
\usepackage{subcaption}
\usepackage{listings,jvlisting}
\usepackage{url}
% 数式
\usepackage{amsmath,amsfonts}
\usepackage{bm}
% 画像
\usepackage[dvipdfmx]{graphicx}
\lstset{
  stringstyle={	tfamily},
  commentstyle={	tfamily},
  basicstyle={	tfamily},
  columns=fixed,
  frame={tb},
  breaklines=true,
  columns=[l]{fullflexible},
  numbers=left,%行数を表示したければonにする
  numberstyle={\scriptsize},
  xrightmargin=0em,
  xleftmargin=3em,
  stepnumber=1,
  numbersep=1em,
  tabsize=2,
  lineskip=-0.5ex,
  backgroundcolor=\color{white}
}
\usepackage{geometry}
\geometry{left=25mm,right=25mm,top=20mm,bottom=20mm}
\usepackage{enumitem}
\usepackage{float}
\usepackage{tikz}
\usetikzlibrary{shapes.geometric, arrows,positioning}
\tikzstyle{startstop} = [rectangle, rounded corners, minimum width=3cm, minimum height=1cm,text centered, draw=black, fill=red!20]
\tikzstyle{process0} = [rectangle, minimum width=3cm, minimum height=1cm, text centered, draw=black, fill=red!20]
\tikzstyle{process1} = [rectangle, minimum width=3cm, minimum height=1cm, text centered, draw=black, fill=blue!20]
\tikzstyle{process2} = [rectangle, minimum width=3cm, minimum height=1cm, text centered, draw=black, fill=orange!30]
\tikzstyle{process3} = [rectangle, minimum width=3cm, minimum height=1cm, text centered, draw=black, fill=green!30]
\tikzstyle{arrow} = [thick,->,>=stealth]
\tikzstyle{io} = [trapezium, trapezium left angle=70, trapezium right angle=110, minimum width=3cm, minimum height=1cm, text centered, draw=black, fill=pink!40]
\title{日本史後期中間まとめ}
\date{\empty}
\author{橋本 千聡}
\begin{document}
\maketitle
\section*{テスト範囲}
\begin{itemize}
  \item 教科書P167~P218
  \item プリント
  \begin{itemize}
    \item 都市と商業・手工業
    \item 幕政の改革
    \item 幕府の衰退と近代への道
  \end{itemize}
\end{itemize}

\section*{出題形式}
\begin{itemize}
  \item 呉軍問題(60点)
  \item 正誤問題(30点)
  \item 筆記問題(10点)
\end{itemize}

\newpage

\section*{都市と商業・手工業}
\subsection*{村と百姓}
\subsubsection*{(1) 村の運営}
\begin{itemize}
  \item 本百姓: 検地帳に登録、年貢・諸役を負担、村政に参加。
  \item 村方三役 (名主・組頭・百姓代): 村政の指導者、本百姓から選ばれることが多い。
\end{itemize}

\subsubsection*{(2) 百姓の負担}
\begin{itemize}
  \item 本途物成 (本年貢): 田畑・屋敷地に課税、米納が原則
  \item 小物成: 山野河海の利用、農業以外の副業に課税
\end{itemize}

\subsubsection*{(3) 百姓の統制}
\begin{itemize}
  \item 村請制: 年貢・諸役の負担を村全体に割り当てる制度
  \item 五人組: 年貢納入、犯罪防止に対する連帯責任制度
\end{itemize}

\subsection*{幕政の安定}
\subsubsection*{4代将軍 徳川家綱〔在 1651~1680〕}
\begin{itemize}
  \item 武断政治から文治政治への転換
  \item 末期養子の禁の緩和: 50歳未満の大名が死の間際にとる養子を容認
  \item 殉死の禁止: 主人の死に殉じる戦国の遺風を廃止
\end{itemize}

\subsubsection*{5代将軍 徳川綱吉〔在 1680~1709〕}
\begin{itemize}
  \item 文治主義の徹底
  \item 武家諸法度(天和令)を発布(1683): 「文武忠孝を励まし、礼儀を正すべき事」→以前は「文武弓馬」だった
  \item 儒学の奨励: 朱子学者・木下順庵に学び、湯島聖堂を建立
\end{itemize}

\subsubsection*{6代将軍 徳川家宣〔在 1709~1712〕}
\begin{itemize}
  \item 朱子学者・新井白石を登用し、政治を刷新
  \item 7代将軍 徳川家継〔在 1713~1716〕の治世を含めて、正徳の治という
  \item 朝幕関係の改善: 閑院宮家の創設
  \item 正徳小判鋳造: 貨幣価値を上げ、物価の抑制をねらう
\end{itemize}

\subsection*{経済の発展}
\subsubsection*{農具}
\begin{itemize}
  \item 備中鍬、千歯扱、唐箕
\end{itemize}

\subsubsection*{肥料}
\begin{itemize}
  \item 干鰯・油粕などの金肥
\end{itemize}

\subsubsection*{産業}
\begin{itemize}
  \item 入浜塩田、西陣織
\end{itemize}

\subsubsection*{交通}
\begin{itemize}
  \item 五街道 (1東海道、2中山道、3甲州道中、4日光道中、5奥州道中)
  \item 菱垣廻船と樽廻船、東廻り海運(東北日本海側~津軽海峡~那珂湊~江戸)と西廻り海運(東北日本海側~下関~大坂)
\end{itemize}

\subsubsection*{貨幣}
\begin{itemize}
  \item 三貨 (金・銀・銭)を幕府が鋳造 「江戸の金遣い、大坂の銀遣い」
\end{itemize}

\subsubsection*{天下の台所}
\begin{itemize}
  \item 大坂には諸藩の蔵屋敷が密集、蔵元・掛屋が活躍
  \item 三都 (江戸・大坂・京都)のうちの一つ
\end{itemize}

\end{document}