\documentclass[a4paper,12pt]{article}
\usepackage[utf8]{inputenc}
\usepackage{luatexja} % LuaLaTeX用日本語パッケージ
\usepackage{geometry}
\geometry{left=25mm,right=25mm,top=20mm,bottom=20mm}

\title{令和6年度 情報工学実験Ⅰ 報告書}
\date{}

\begin{document}

\maketitle

\section*{実験題目}
\underline{プログラミング演習4}

\section*{指導教員}
\underline{丸山教員,安細教員,周教員}

\section*{実験日}
\begin{itemize}
    \item 令和6年 10月 2日 (水)
    \item 令和6年 10月 16日 (水)
\end{itemize}

\section*{レポート}
\begin{itemize}
    \item 提出締切日: 令和6年 10月 30日 (水)
    \item 受理最終日: 令和6年 11月 20日 (水)
    \item 提出日: 令和6年 \underline{\hspace{1cm}} 月 \underline{\hspace{1cm}} 日 (\underline{\hspace{1cm}})
\end{itemize}

\section*{報告者}
2年 31番 氏名 \underline{橋本 千聡}

\section*{共同実験者}
川和 李圭, 鈴木 隆生, 安田 れん

\newpage

\section*{1. 実験の目的}
プログラムの共同開発演習を通して、議論などを伴うチームでのプログラム作成手法を理解する。

\section*{2. 実験の概要}
\begin{itemize}
    \item 1・2週目: 実行環境の確認及びC言語サンプル実行確認 \\
    作成分担調整、分担一覧や全体構成の資料作成
    \item 3・4週目: 各自の担当箇所を作成、単体動作確認
    \item 5・6週目: 各自の作成の関数を統合して動作確認、レポート報告内容のまとめ、レポート作成
\end{itemize}

\section*{3. 演習課題の報告}
\begin{itemize}
    \item プログラム全体の概要
    \item 担当した機能の構成と説明
    \item 担当した機能の単体テストの方法と結果
    \item 統合したプログラムの結合テストの方法と結果
\end{itemize}

\section*{4. 実験の感想}
% ここに感想を記入してください
\end{document}
