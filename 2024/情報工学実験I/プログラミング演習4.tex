\documentclass[lualatex, ja=standard, a4j, 12pt]{bxjsarticle}
\usepackage{luatexja-fontspec}
\setmainjfont{IPAexGothic}

\usepackage{geometry}
\geometry{left=25mm,right=25mm,top=20mm,bottom=20mm}
\usepackage{enumitem}
\usepackage{float}
\usepackage{tikz}
\usetikzlibrary{shapes.geometric, arrows}

\tikzstyle{startstop} = [rectangle, rounded corners, minimum width=3cm, minimum height=1cm,text centered, draw=black, fill=red!30]
\tikzstyle{process} = [rectangle, minimum width=3cm, minimum height=1cm, text centered, draw=black, fill=orange!30]
\tikzstyle{decision} = [diamond, minimum width=3cm, minimum height=1cm, text centered, draw=black, fill=green!30]
\tikzstyle{arrow} = [thick,->,>=stealth]

\title{\underline{令和6年度 情報工学実験I 報告書}}
\date{}

\begin{document}
\maketitle
\section*{実験題目}
\underline{プログラミング演習4}

\section*{指導教員}
\underline{丸山教員,安細教員,周教員}

\section*{実験日}
\begin{itemize}[left=2em]
    \item \underline{令和6年 10月 02日 (水)〜令和6年 10月 16日 (水)}
\end{itemize}

\section*{レポート}
\begin{itemize}[left=2em]
    \item \underline{提出締切日: 令和6年 10月 30日 (水)}
    \item \underline{受理最終日: 令和6年 11月 20日 (水)}
    \item \underline{提出日: 令和6年 \underline{\hspace{1cm}} 月 \underline{\hspace{1cm}} 日 (\underline{\hspace{1cm}})}
\end{itemize}

\section*{報告者}
\underline{2年 31番 氏名 橋本 千聡}

\section*{共同実験者}
\underline{川和 李圭, 鈴木 隆生, 安田 れん}

\newpage


\section*{1. 実験の目的}
プログラムの共同開発演習を通して、議論などを伴うチームでのプログラム作成手法を理解する。

\section*{2. 実験の概要}
\begin{itemize}[left=2em]
    \item 1・2週目: 実行環境の確認及びC言語サンプル実行確認 \\
    作成分担調整、分担一覧や全体構成の資料作成
    \item 3・4週目: 各自の担当箇所を作成、単体動作確認
    \item 5・6週目: 各自の作成の関数を統合して動作確認 \\
    レポート報告内容のまとめ、レポート作成
\end{itemize}

\section*{3. 演習課題の報告}
プログラム全体の概要
\begin{figure}[H]
    \centering
    \begin{tikzpicture}[node distance=2cm]
        \node (start) [startstop] {開始};
        \node (init) [process, below of=start] {初期化};
        \node (input) [process, below of=init] {入力処理};
        \node (decision) [decision, below of=input, yshift=-1cm] {条件判定};
        \node (process1) [process, below of=decision, yshift=-1cm] {処理1};
        \node (process2) [process, right of=decision, xshift=3cm] {処理2};
        \node (end) [startstop, below of=process1] {終了};

        \draw [arrow] (start) -- (init);
        \draw [arrow] (init) -- (input);
        \draw [arrow] (input) -- (decision);
        \draw [arrow] (decision) -- node[anchor=east] {Yes} (process1);
        \draw [arrow] (decision) -- node[anchor=south] {No} (process2);
        \draw [arrow] (process1) -- (end);
        \draw [arrow] (process2) |- (end);
    \end{tikzpicture}
    \caption{プログラムのフローチャート}
    \label{fig:flowchart}
\end{figure}

担当した機能の構成と説明

担当した機能の単体テストの方法と結果

統合したプログラムの結合テストの方法と結果

\section*{4. 実験の感想}
% ここに感想を記入してください
\end{document}
