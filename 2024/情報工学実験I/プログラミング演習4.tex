\RequirePackage{plautopatch}
\documentclass[a4paper,11pt]{jsarticle}

\usepackage{color}
\usepackage{subcaption}
\usepackage{listings,jvlisting}
\usepackage{url}
% 数式
\usepackage{amsmath,amsfonts}
\usepackage{bm}
% 画像
\usepackage[dvipdfmx]{graphicx}

\lstset{
  stringstyle={\ttfamily},
  commentstyle={\ttfamily},
  basicstyle={\ttfamily},
  columns=fixed,
  frame={tb},
  breaklines=true,
  columns=[l]{fullflexible},
  numbers=left,%行数を表示したければonにする
  numberstyle={\scriptsize},
  xrightmargin=0em,
  xleftmargin=3em,
  stepnumber=1,
  numbersep=1em,
  tabsize=2,
  lineskip=-0.5ex,
  backgroundcolor=\color{white}
}
\usepackage{geometry}
\geometry{left=25mm,right=25mm,top=20mm,bottom=20mm}
\usepackage{enumitem}
\usepackage{float}
\usepackage{tikz}
\usetikzlibrary{shapes.geometric, arrows,positioning}

\tikzstyle{startstop} = [rectangle, rounded corners, minimum width=3cm, minimum height=1cm,text centered, draw=black, fill=red!20]
\tikzstyle{process0} = [rectangle, minimum width=3cm, minimum height=1cm, text centered, draw=black, fill=red!20]
\tikzstyle{process1} = [rectangle, minimum width=3cm, minimum height=1cm, text centered, draw=black, fill=blue!20]
\tikzstyle{process2} = [rectangle, minimum width=3cm, minimum height=1cm, text centered, draw=black, fill=orange!30]
\tikzstyle{process3} = [rectangle, minimum width=3cm, minimum height=1cm, text centered, draw=black, fill=green!30]
\tikzstyle{arrow} = [thick,->,>=stealth]
\tikzstyle{io} = [trapezium, trapezium left angle=70, trapezium right angle=110, minimum width=3cm, minimum height=1cm, text centered, draw=black, fill=pink!40]



\title{\underline{令和6年度 情報工学実験I 報告書}}
\date{\empty}
\author{\empty}

\begin{document}
\maketitle
\section*{実験題目}
\underline{プログラミング演習4}

\section*{指導教員}
\underline{丸山教員,安細教員,周教員}

\section*{実験日}
\begin{itemize}[left=2em]
    \item \underline{令和6年 10月 02日 (水)〜令和6年 10月 16日 (水)}
\end{itemize}

\section*{レポート}
\begin{itemize}[left=2em]
    \item \underline{提出締切日: 令和6年 10月 30日 (水)}
    \item \underline{受理最終日: 令和6年 11月 20日 (水)}
    \item \underline{提出日: 令和6年 \underline{\hspace{1cm}} 月 \underline{\hspace{1cm}} 日 (\underline{\hspace{1cm}})}
\end{itemize}

\section*{報告者}
\underline{2年 31番 氏名 橋本 千聡}

\section*{共同実験者}
\underline{川和 李圭, 鈴木 隆生, 安田 れん}

\newpage


\section*{1. 実験の目的}
プログラムの共同開発演習を通して、議論などを伴うチームでのプログラム作成手法を理解する。

\section*{2. 実験の概要}
\begin{itemize}[left=2em]
    \item 1・2週目: 実行環境の確認及びC言語サンプル実行確認 \\
    作成分担調整、分担一覧や全体構成の資料作成
    \item 3・4週目: 各自の担当箇所を作成、単体動作確認
    \item 5・6週目: 各自の作成の関数を統合して動作確認 \\
    レポート報告内容のまとめ、レポート作成
\end{itemize}


\section*{3. 演習課題の報告}
\subsection*{プログラム全体の概要}
\begin{figure}[H]
    \centering
    \begin{tikzpicture} [node distance=2.5cm]
        \node (main)  [startstop, minimum width=2cm] {main関数};
        \node (sort) [process0, below of=main, minimum width=2cm] {sort};
        \node (load)  [process0, left of=sort, minimum width=2cm] {load};
        \node (save)  [process0, below of=load, minimum width=2cm] {save};
        \node (input)  [process1, right of=sort, minimum width=2cm] {input};
        \node (search)[process2, right of=input, minimum width=2cm] {search};
        \node (show)  [process0, right of=search, minimum width=2cm] {show};
        \node (country)[process3, right=1cm of show, minimum width=2cm] {show\_country};
        \node (rank)  [process2, below of=country, minimum width=2cm] {get\_medalrank};
        \node (data)  [io, below of=search, minimum width=2cm] {データ構造};
        \node (csv)   [io, below of=data, minimum width=2cm] {CSVファイル};
        \node (term)  [io, above of=show, minimum width=2cm] {ターミナル};
        \node (len0202)[process2, above of=main, minimum width=2cm] {安田};
        \node (kazu321)[process0, left of=len0202, minimum width=2cm] {橋本};
        \node (kawawarika)[process1, right of=len0202, minimum width=2cm] {川和};
        \node (ryuusei899)[process3, right of=kawawarika, minimum width=2cm] {鈴木};
        \draw [arrow] (main) -- (input);
        \draw [arrow] (main) -- (load);
        \draw [arrow] (main) -- (save);
        \draw [arrow] (main) -- (sort);
        \draw [arrow] (main) -- (search);
        \draw [arrow] (main) -- (show);
        \draw [arrow] (main) -- (country);
        \draw [arrow] (data) -- (rank);
        \draw [arrow] (input)-- (data);
        \draw [arrow] (load) -- (data);
        \draw [arrow] (data) -- (save);
        \draw [arrow] (data) -- (sort);
        \draw [arrow] (sort) -- node{上書き} (data);
        \draw [arrow] (data) -- (search);
        \draw [arrow] (data) -- (show);
        \draw [arrow] (csv)  -- node[xshift=3cm, yshift=-1.2cm]{読み込み} (load);
        \draw [arrow] (save) -- node[xshift=-0.5cm,yshift=-0.3cm]{書き込み} (csv);
        \draw [arrow] (main) -- node[yshift=0.5cm]{コマンド入出力} (term);
        \draw [arrow] (term) -- (main);
        \draw [arrow] (term) -- node[xshift=-0.1cm,yshift=0.4cm]{入力} (search);
        \draw [arrow] (term) -- (input);
        \draw [arrow] (show) -- node[anchor=west]{表示} (term);
        \draw [arrow] (country) -- (show);
        \draw [arrow] (sort) edge[bend right,out=10] (country);
    \end{tikzpicture}
    \caption{プログラムのフローチャート+$\alpha$}
    \label{fig:flowchart}
\end{figure}

\subsection*{担当した機能の構成と説明}

\subsubsection*{sort関数}
\begin{itemize}
    \item 関数の説明: データ構造を並び替える
    \item 関数の入力: モード
    \item 関数の出力: なし(データ構造上書き)
    \item 関数の処理内容: merge sortを行い、modeによって国名金銀銅総メダル数のいずれかで並び替える
\end{itemize}

\begin{lstlisting}[caption=sort関数のコード, label=sort, language=C]
    #include "main.h"

    // 安定な比較関数
    int compare_by_mode(const country_data_type* a, const country_data_type* b, int mode) {
        if (mode == 0) {
            return strcmp(a->country, b->country);   // 国名で昇順
        } else if (mode == 1) {
            return b->gold - a->gold;   // 金メダル数で降順
        } else if (mode == 2) {
            return b->silver - a->silver;   // 銀メダル数で降順
        } else if (mode == 3) {
            return b->bronze - a->bronze;   // 銅メダル数で降順
        } else if (mode == 4) {
            return b->sum - a->sum;   // 合計メダル数で降順
        }
        return 0;  // デフォルトは等しいと見なす
    }
    
    // 安定なマージ関数
    void merge(country_data_type arr[], int left, int mid, int right, int mode) {
        int n1 = mid - left + 1;
        int n2 = right - mid;
        country_data_type L[n1], R[n2];
        for (int i = 0; i < n1; i++)
            L[i] = arr[left + i];
        for (int j = 0; j < n2; j++)
            R[j] = arr[mid + 1 + j];
    
        int i = 0, j = 0, k = left;
    
        while (i < n1 && j < n2) {
            if (compare_by_mode(&L[i], &R[j], mode) <= 0) {
                arr[k] = L[i];
                i++;
            } else {
                arr[k] = R[j];
                j++;
            }
            k++;
        }
    
        while (i < n1) {
            arr[k] = L[i];
            i++;
            k++;
        }
    
        while (j < n2) {
            arr[k] = R[j];
            j++;
            k++;
        }
    }
    
    // マージソート関数
    void mergeSort(country_data_type arr[], int left, int right, int mode) {
        if (left < right) {
            int mid = left + (right - left) / 2;
            mergeSort(arr, left, mid, mode);
            mergeSort(arr, mid + 1, right, mode);
            merge(arr, left, mid, right, mode);
        }
    }
    
    // ソート関数
    void sort(int mode) {
        mergeSort(data, 0, data_size - 1, mode);
    }
\end{lstlisting}

\subsubsection*{load関数}
\begin{itemize}
    \item 関数の説明: CSVファイルからデータを読み込む
    \item 関数の入力: ファイル名
    \item 関数の出力: なし(データ構造上書き)
    \item 関数の処理内容: 引数のファイル名からCSVファイルを読み込み、データ構造に格納する
\end{itemize}
\begin{lstlisting}[caption=load関数のコード, label=load, language=C]
#include "main.h"

void load(char* filename){
    FILE *fp;
    char full_filename[256];
    snprintf(full_filename, sizeof(full_filename), "./data/%s.csv", filename);
    printd("load file: %s\n", full_filename);
    fp = fopen(full_filename, "r");
    if(fp == NULL){
        printf("ファイルが開けません\n");
        return;
    }
    printd("load data start\n");
    char buf[4][100];
    fscanf(fp, "%[^,],%[^,],%[^,],%s\n", buf[0], buf[1], buf[2], buf[3]);
    printd("header: %s, %s, %s, %s\n", buf[0], buf[1], buf[2], buf[3]);
    data_size = 0;
    while(fscanf(fp, "%[^,],%d,%d,%d\n", data[data_size].country, &data[data_size].gold, &data[data_size].silver, &data[data_size].bronze) != EOF){
        printd("load data[%d]: %s %d %d %d %d\n", data_size, data[data_size].country, data[data_size].gold, data[data_size].silver, data[data_size].bronze, data[data_size].medal_rank);
        data_size++;
    }
    get_medalrank(data_size);
    get_sum();
    fclose(fp);
}
\end{lstlisting}

\subsubsection*{save関数}
\begin{itemize}
    \item 関数の説明: データ構造をCSVファイルに書き込む
    \item 関数の入力: ファイル名
    \item 関数の出力: なし
    \item 関数の処理内容: 引数のファイル名にデータ構造を書き込む
\end{itemize}


\subsubsection*{show関数}
\begin{itemize}
    \item 関数の説明: データ構造を表示する
    \item 関数の入力: なし
    \item 関数の出力: なし
    \item 関数の処理内容: printfのフォーマット機能を駆使してテーブル形式でデータ構造を表示する
\end{itemize}

\subsubsection*{main関数}
\begin{itemize}
    \item 関数の説明: メイン関数
    \item 関数の入力: 引数DEBUGの有無
    \item 関数の出力: 標準出力
    \item 関数の処理内容: bashをベースとしたターミナルでのコマンド入力を受け付け、各関数を呼び出す
    \item その他: デバッグモードを有効にすると、各関数の詳細情報を表示する
\end{itemize}

\subsection*{担当した機能の単体テストの方法と結果}

\subsection*{統合したプログラムの結合テストの方法と結果}

\section*{4. 実験の感想}
% ここに感想を記入してください
\end{document}
